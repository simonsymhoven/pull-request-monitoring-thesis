\chapter{Abstract}
\label{Abstract}

Das Ziel in der vorliegenden Arbeit ist es, ein neues \textit{Dashboard} in Form eines \textit{Plugins} für den \acs{ci}-Server \textit{Jenkins} zu entwickeln. Das \textit{Plugin} soll ein für den Nutzer konfigurierbares und beliebig erweiterbares \textit{Dashboard} zur Verfügung stellen, welches verschiedene Delta-Softwaremetriken von \textit{Pull Requests} visualisiert. Anders als bisherige Darstellungen innerhalb des \textit{Jenkins} \acs{ui} sollen die Resultate eines \textit{Pull Requests} auf eine gefilterte Darstellung der tatsächlich vorgenommenen Änderungen an der Codebasis reduziert werden. 
 
Das \textit{Pull Request Monitoring Plugin} wurde auf Basis der \textit{JavaScript} Bibliothek \textit{Muuri} entwickelt und bietet eine Schnittstelle für andere \textit{Plugins}, ihre Metriken und Qualitätsmerkmale dem \textit{Dashboard} in Form eines \textit{Portlets} zur Verfügung zu stellen.

Es werden die technische Umsetzung des \textit{Plugins} erläutert, Entscheidungen diskutiert und resultierende Ergebnisse vorgestellt. Dabei wird auf die zwei umgesetzten \textit{Portlets} des \textit{Warnings Next Generation} und \textit{Code Coverage \acs{api}} \textit{Plugins} eingegangen und die für den Anwender relevante grafische Benutzer\-oberfläche des \textit{Plugins} innerhalb des \textit{Jenkins} \acs{ui} präsentiert. 
