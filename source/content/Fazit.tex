\chapter{Zusammenfassung und Ausblick}
\label{chap:fazit}

Statt die Resultate für das Gesamtprojekt zu visualisieren, ist eine gefilterte Darstellung der Ergebnisse auf die tatsächlich vorgenommene Änderung an der Codebasis innerhalb eines \textit{Pull Requests} sinnvoll. Diese Arbeit beschäftigte sich mit der Konzeptionierung und Entwicklung eines \textit{Jenkins} \textit{Plugins} zur Überwachung bestimmter Qualitätskriterien und Metriken eines \textit{Pull Requests}. Diese beziehen sich auf die durch den \textit{Pull Request} veränderte Codebasis - das Delta zwischen Quell- und Ziel-\textit{Branch}. Das Monitoringsystem wird dem Nutzer in Form eines konfigurierbaren und eines durch andere \textit{Plugins} erweiterbaren \textit{Dashboards} innerhalb des \textit{Jenkins} \ac{ui} zur Verfügung gestellt. Die grundlegenden Funktionen des \textit{Dashboards} basieren dabei auf der externen Bibliothek \textit{Muuri}. Die Konfiguration des \textit{Dashboards} wird browser- und rechnerunabhängig pro Nutzer innerhalb des \textit{Jenkins} gespeichert und ist jederzeit abrufbar. Außerdem wurde ein \ac{api} entwickelt, welches andere \textit{Plugins} konsumieren und ihre Metriken als \textit{Portlets} dem Nutzer bereitstellen können. Diese können somit in dem nutzerspezifischen \textit{Dashboard} hinzugefügt und überwacht werden. Dieses \ac{api} wurde im Rahmen der vorliegenden Arbeit verwendet, um die ersten \textit{Portlets} in dem \textit{Warnings Next Generation} und \textit{Code Coverage \ac{api} Plugin} zu entwickeln und zu veröffentlichen. Das \textit{Pull Request Monitoring Plugin} ist unter der \textit{MIT}-Lizenz als \textit{Open Source} Projekt veröffentlicht und 
steht zur Installation innerhalb des \textit{Jenkins} Servers bereit. 

Abschließend werden diejenigen Punkte angesprochen, für die die Zeit nicht ausgereicht hat, die aber im Rahmen weiterer Studien oder Erweiterungen des \textit{Plugins} interessant wären zu untersuchen und umzusetzen.

Die Größe eines \textit{Portlets} wird in Pixeln von dem implementierenden \textit{Plugin} festgelegt und verändert diese Größe nicht, wenn das \textit{Dashboard}, respektive das Browserfenster verkleinert oder vergrößert wird. Denkbar ist es, dies durch ein \textit{Grid}-System zu ersetzen, sodass die Größe der \textit{Portlets} sich der Größe des Browserfensters anpassen, statt eine statische Größe zu definieren.

Des Weiteren können die \textit{Portlets} innerhalb des \textit{Dashboards} nicht bearbeitet werden. Dazu müssen diese zuerst entfernt werden und dann über den Dialog aus \autoref{fig:add2} mit den neuen Parametern dem \textit{Dashboard} wieder hinzugefügt werden. Praktikabler wäre eine Möglichkeit für den Nutzer, die bereits existierenden \textit{Portlets} über einen weiteren Dialog direkt bearbeiten zu können ohne diese vorher löschen zu müssen. 

Ebenfalls sollen für weitere \textit{Plugins} wie dem \textit{JUnit} \citep{junit-plugin} oder dem \textit{Git Forensics} \textit{Plugin} \citep{git-forensics-plugin} \textit{Portlets} entwickelt werden und weitere Autoren motiviert werden, die zur Verfügung gestellte \ac{api}  zu nutzen, um der \textit{Jenkins Community} ein vielfältigeres \textit{Dashboard} bereitstellen zu können. 

Durch die Vorstellung der ersten Beta-Version des \textit{Plugins} in einem der \textit{Jenkins UX SIG} Treffen, wurde die Comquent GmbH aus Puchheim auf das \textit{Plugin} und diese Arbeit aufmerksam.  Dadurch ergab sich die Möglichkeit, die vorliegende Arbeit auch im Kontext der wirtschaftlichen Anwendung zu diskutieren und voraussichtlich in aktiven Projekten bei Kunden der Comquent GmbH einzusetzen. Dadurch wäre es besonders interessant herauszufinden, welche Anforderungen sich aus der wirtschaftlichen Anwendung ergeben, um diese mit in die Weiterentwicklung des \textit{Plugins} einfließen lassen zu können. In einem ersten Gespräch mit dem Geschäftsführer der Comquent GmbH, Herrn Andreas Schönfeld, wies dieser darauf hin, dass sich die Anforderungen der Wirtschaft nicht selten von denen aus der Entwickler-Szene unterscheiden, da Trends, Prozesse und Arbeitsweisen häufig sehr viel später antizipiert werden. 
Als mögliche Weiterentwicklung des \textit{Plugins} wurden die Einflussfaktoren der Delta-Resultate diskutiert: Welcher Entwickler hat durch welche Änderungen welche Metriken wie verändert? 
Aufgrund der zeitlichen Einschränkungen können die Ergebnisse der praktischen Anwendung nicht in der vorliegenden Arbeit diskutiert werden. 
Besonders bedanke ich mich bei Herrn Andreas Schönfeld für die Kooperation und die Zusammenarbeit. Diese hat mir wertvolle Einblicke gewährt und eine praxisnahe Forschung ermöglicht, die über diese Arbeit hinaus anhalten wird.