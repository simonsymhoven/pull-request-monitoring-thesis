\chapter{Verwandte Arbeiten}
\label{chap:verwandte-arbeiten}

Es gibt bereits zahlreiche Plugins für den \ac{ci}-Server \textit{Jenkins}, die Softwaremetriken berechnen. Darunter das \textit{Code Coverage API Plugin}, welches die \textit{Code Coverage} einer Codebasis ermittelt \citep{code-coverage-api-plugin}, das \textit{Warnings Next Generation Plugin}, welches Compiler-Warnungen sammelt, die von statischen Analysetools gemeldet werden \citep{warnings-ng-plugin}, oder das \textit{Git Forensics API Plugin}, welches Daten aus einem Versionskontroll-\textit{Repository} ausliest und analysiert \citep{forensics-api-plugin}. 
Jedes \textit{Plugin} bietet eine eigene Darstellung ihrer Ergebnisse. Meist werden die Metriken jedoch nur bezogen auf das Gesamtprojekt berechnet und nicht speziell für \textit{Pull Requests}. 

Erste Ansätze eines dedizierten \textit{Dashboards}, welches Ergebnisse anderer \textit{Plugins} aggregiert, sind in dem \textit{Dashboard View Plugin} zu erkennen. Es können \textit{Views} angelegt werden und diese mit Ergebnissen anderer \textit{Plugins} befüllt werden, sofern diese ihre Ergebnisse in dem entsprechenden Format bereitstellen. Die \textit{Portlets}, die von anderen \textit{Plugins} bereitgestellt werden, beziehen sich aber nicht zwangsläufig auf reine Softwaremetriken oder gar die Darstellung einzelner Resultate für einen \textit{Pull Request} \citep{dashboard-view-plugin}. Vielmehr ist dieses \textit{Plugin} im Kontext des Gesamtprojekts zu betrachten und dient der generellen Überwachung einzelner Projekte. Damit wurde das Problem adressiert, das historisch gewachsene \ac{ui} des \textit{Jenkins} zu modernisieren und diverse Resultate in einem \textit{Dashboard} zu aggregieren. Es fehlt der Bezug  zum \textit{Pull Request} und die Reduzierung der Resultate auf die durch den \textit{Pull Request} veränderten Codebestandteile. 