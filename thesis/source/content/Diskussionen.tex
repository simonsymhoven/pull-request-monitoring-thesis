\chapter{Diskussionen}
\label{chap:diskussionen}

\section{Allgemeine Voraussetzung}
Kernaufgabe des \textit{Jenkins \ac{ci}}-Servers ist die  Ausführungen von \textit{Jobs}. Es gibt \textit{Freestyle-Jobs}, in denen die einzelnen \textit{Build Steps} konfigurierbar definiert werden. \textit{Pipelines} sind spezielle \textit{Jobs}, deren Definition in Form von \textit{Groovy}-Code erfolgt. Diese können in der \textit{Job}-Konfiguration oder im \ac{scm} abgelegt werden. Durch die Ablage im \ac{scm} und der Verwendung des \textit{Plugins} \textit{Pipeline: Mutlibranch} wird das Erkennen von verschiedenen \textit{Branches} in dem verwendeten \ac{scm}-\textit{Repository} ermöglicht. Jeder \textit{Branch} wird als ein separater \textit{Job} behandelt, für den unabhängige Arbeitsabläufe als \textit{Pipeline} oder innerhalb der \textit{Job} Konfiguration definiert werden können \citep{workflow-multibranch-plugin}. 
Das \textit{Pull Request Monitoring Plugin} beschränkt sich auf die Verwendung von \textit{Multibranch} Projekten. 


\section{\textit{Abstract Class} statt \textit{Interface}}
Mit Version \textit{1.6.0} wurden die beiden Klassen \textit{MonitorPortlet} und \textit{MonitorPortletFactory} von einem \textit{Interface} zu einer \textit{Abstract Class} geändert \citep{pull-request-monitoring-plugin}. Dies bietet den Vorteil, dass bestimmte Methoden mit einer Implementierung belegt werden können, die nur bei Bedarf von der erbenden Klasse überschrieben werden müssen. Des Weiteren bietet die Verwendung einer \textit{Abstract Class} eine deutlich stabilere Schnittstelle nach außen an, da im Falle von einer Erweiterung eine vordefinierte Implementierung der hinzugefügten Methode mitgeliefert werden kann, sodass es bei der erbenden Klasse zu keinen Fehlern während des Kompilierens kommt.

\section{\textit{User Property} statt \textit{Local Storage}}

Bis Version \textit{1.0.3-beta} wurde die Konfiguration des \textit{Dashboards}, also der im \textit{Dashboard} verwendeten \textit{Portlets} im lokalen Speicher, dem \textit{Local Storage} des Browsers persistiert. Darunter die Breite, Höhe und die eindeutige ID jedes \textit{Portlets}, um diese beispielsweise bei der Aktualisierung der Seite erneut zu laden \citep{pull-request-monitoring-plugin}. Diese erste lauffähige Beta-Version wurde im \textit{Jenkins UX SIG Meeting}, dem \textit{Jenkins User Experience Special Interest Group Meeting} präsentiert \citep{yt}. Tim Jacomb, ein aktives Mitglied der \textit{UX SIG} Gruppe, merkte an, dass das Speichern der Konfiguration im lokalen Speicher des Browsers diverse Nachteile mit sich bringt. Konfigurationen sind somit abhängig vom eingesetzten Browser, können also beim Verwenden eines anderen Browsers nicht übernommen werden. Dasselbe gilt, wenn Nutzer sich an unterschiedlichen Rechnern anmelden, um auf eine \textit{Jenkins} Instanz zuzugreifen. Er schlug vor die Konfiguration stattdessen in einer \textit{User Property}, der \textit{MonitorConfigurationProperty} innerhalb der \textit{Jenkins} Infrastruktur zu speichern, um zu jederzeit browser- und rechnerunabhängig auf die Konfiguration zugreifen zu können. 
 
\section{Auswahl der \textit{JavaScript} Bibliothek \textit{Muuri}}

Es existieren diverse \textit{JavaScript} Bibliotheken, welche bereits ein konfigurierbares, verschiebbares und dynamisches Layout für ein \textit{Dashboard} zur Verfügung stellen, darunter \textit{gridstack.js} \citep{gridstack} oder \textit{Muuri}. \textit{Muuri} stellte sich als für diesen Zweck einzig brauchbare Bibliothek heraus, da es diese dem Entwickler erlaubt, den Inhalt der einzelnen \textit{Items}, den \textit{Portlets} als HTML zu setzen und über Klassenzuweisungen dem \textit{Dashboard} hinzuzufügen und zu steuern.
Bei \textit{gridstack.js} ist dies nicht möglich. Hier muss der zu setzende Inhalt statisch über das \textit{API} der Bibliothek gesetzt werden. Das stellte sich als nicht praktikabel heraus, da die bereitgestellten \textit{Portlets} als \textit{Jelly-View} ausgeliefert werden. Diese können nur in anderen \textit{Jelly-Views} inkludiert werden und nicht im Nachhinein hinzugefügt werden, da das \textit{Jelly} Rendering serverseitig beim Aufruf der entsprechenden \ac{url} initiiert wird.

\section{\textit{Default}-\textit{Portlets} für das \textit{Dashboard}}
\label{chap:default}

Seit Version 1.7.0 kann ein \textit{Portlet} entscheiden, ob dieses standardmäßig im \textit{Dashboard} angezeigt werden soll \citep{pull-request-monitoring-plugin}. Sofern für das jeweilige \textit{Dashboard} keine nutzerspezifische Konfiguration existiert, werden dem Nutzer beim Öffnen eines zu einem \textit{Run} zugehörigen \textit{Dashboards} alle vorhanden \textit{Portlets} angezeigt, die als \textit{default} markiert sind. Das soll dem Nutzer den Einstieg erleichtern und die manuelle Konfiguration der \textit{Dashboards} vermeiden. Durch lokale Änderungen über das \ac{ui} des \textit{Dashboards} oder die Anwendung einer im \textit{Jenkinsfile} definierten Konfiguration, wird diese \textit{default}-Konfiguration überschrieben. 
