\chapter{Einleitung}
\label{chap:einleitung}

Die \acf{ci} hat in den letzten Jahren enorm an Bedeutung gewonnen. Viele Software Teams haben ihre Entwicklungsprozesse auf \textit{Pull Requests} umgestellt. Änderungen an der Software werden auf \textit{Feature Branches} unter Anwendung eines \textit{Git-Workflows} entwickelt und in diesem mithilfe eines \acf{scm} Systems in dem zugehörigen \textit{Repository} eingecheckt, der dann durch einen \textit{Pull Request} in die bestehende Codebasis integriert wird \citep{fowler_2021}. Dieser wird anschließend manuell einem \textit{Code Review} unterzogen und zusätzlich automatisiert im \ac{ci}-Server, z.\,B. dem \textit{Jenkins}, gebaut \citep{ausschreibung}. Dieser liefert eine Qualitätsaussage zu der jeweiligen Softwareversion.

Seit der Entwicklung des \textit{Jenkins} im Jahre 2004 ist die Visualisierung der Ergebnisse des automatisierten \textit{Builds} oder \textit{Runs} eher im Kontext des Gesamtprojekts zu betrachten. Für Entwickler ist diese Darstellung unzureichend. Statt die Resultate als Übersicht für das Gesamtprojekt zu visualisieren, ist eine gefilterte Darstellung der Ergebnisse auf die tatsächlich vorgenommenen Änderungen durch den \textit{Pull Request} an der Codebasis hilfreicher \citep{ausschreibung}.

Ziel ist es, den \ac{ci}-Server \textit{Jenkins} um ein neues \textit{Dashboard} zu erweitern, welches die Ergebnisse eines \textit{Pull Requests} visualisiert. Wichtige Ergebnisse können sein: Testergebnisse (Fehler, hinzugefügte Tests), Veränderungen der \textit{Code Coverage}, Warnungen einer statischen Code Analyse, \textit{\acs{api} Contract} Verstöße oder Veränderungen der Metriken zur Softwarequalität. 
Das \textit{Dashboard} soll als neues \textit{Plugin}, dem \textit{Pull Request Monitoring Plugin} \citep{pull-request-monitoring-plugin} für den \textit{Jenkins} bereit gestellt werden und eine Möglichkeit bieten, dieses \textit{Dashboard} als Nutzer zu konfigurieren und die Ergebnisse diverser Metriken zu visualisieren. Diese Metriken sollen im Allgemeinen von anderen Plugins bereitgestellt werden. Das \textit{Dashboard} soll erweiterbar sein, sodass andere Plugins eine Möglichkeit haben, ihre Ergebnisse dem \textit{Dashboard} beisteuern zu können. Auch sollen erste \textit{Plugins}, wie das \textit{Warnings Next Generation Plugin} \citep{warnings-ng-plugin} das \ac{api} des erstellten \textit{Pull Request Monitoring Plugins} nutzen können, um die Metriken in Form von \textit{Portlets} dem \textit{Dashboard} und somit dem Nutzer bereitzustellen.
Auf Basis der \textit{JavaScript} Bibliothek \textit{Muuri} soll das \textit{Plugin} entwickelt werden. \textit{Muuri} liefert ein sortierbares, filterbares und verschiebbares Layout \textit{Muuri} \citep{muuri}. Dieses Layout soll mit den bereitgestellten \textit{Portlets} anderer \textit{Plugins} ausgestattet werden, wobei ein \textit{Portlet} eine Komponente in Form einer Kachel mit der darzustellenden Metrik oder Qualitätsaussage innerhalb der Benutzeroberfläche definiert.

\autoref{chap:verwandte-arbeiten} stellt bereits bestehende Arbeiten im wissenschaftlichen Kontext vor. In \autoref{chap:grundlagen} werden dann zunächst die nötigen Grundlagen geschaffen. Dabei werden vor allem auf die im weiteren Verlauf benötigten Begrifflichkeiten eingegangen und diese im Kontext der Arbeit erläutert. Auf die praktische Umsetzung und die damit verbundenen Entscheidungsfindungen wird in \autoref{chap:umsetzung} eingegangen. In \autoref{chap:ergebnisse} werden schließlich die Ergebnisse vorgestellt. \autoref{chap:diskussionen} greift einige Diskussionspunkte auf, welche während der Entwicklung der Arbeit entstanden sind und begründet die getroffenen Entscheidungen. Neben der Zusammenfassung wird in \autoref{chap:fazit} auch ein Ausblick auf mögliche Verbesserungen und Erweiterungen gegeben sowie auf die Zusammenarbeit mit der Comquent GmbH eingegangen.

